\song{Aqu\'{i} esta la Tuna}
\rhythmandkey{Paso doble}{\chord*{A}}
\begin{verse}{Intro}
|bandurria | \chord*{A} \hspace{3em} | \hspace{4em} | \hspace{2em} \chord*{\underline{64 62}} |\\
|\chord*{E} \hspace{3.3em} | \hspace{4em} | \hspace{4em} | \hspace{2em} \chord*{\underline{62 64}} |\\
|\chord*{A} \hspace{3.25em} | \hspace{4em} | \hspace{4em} | \hspace{2em} \chord*{\underline{64 62}} |\\
|\chord*{E} \hspace{3.3em} | \hspace{4em} | \hspace{4em} | \hspace{2em} \chord*{\underline{62 64}} | \chord*{A} \hspace{3em} | \hspace{4em} |\chord*{A}$\downarrow$\\
\end{verse}

\begin{verse}{Couplet 1}
Aqu\'{i} \chord{A}est\'{a} la \chord{E}tuna que con su ale\chord{A}gr\'{i}a,\\
recorre las calles con una can\chord{E}ci\'{o}n.\\
\chord{}Y con sus guitarras, (y con sus guitarras)\\
y con su alegr\'{i}a alegre la vida de la pobla\chord{|\chord*{A} \hspace{3em} | \hspace{4em} | \chord*{A}$\downarrow$}ci\'{o}n.
\end{verse}

\begin{verse}{Couplet 2}
Son \chord{A}los estu\chord{E}diantes, muchachos de \chord{A}broma,\\
de buenas palabras y gran cora\chord{E}z\'{o}n.\\
\chord{}Y son trovadores, (y son trovadores)\\
\chord{}que llevan en notas, palabras, muchachas,\\
un poco de a\chord{A$\downarrow$}mor.\hspace{1em} \chord{E$\downarrow$}\hspace{2em}\chord{A$\downarrow$}
\end{verse}

\begin{verse}{Intermezzo}
|bandurria | \chord*{A} \hspace{3em} | \hspace{4em} | \hspace{2em} \chord*{\underline{64 62}} |\\
|\chord*{E} \hspace{3.3em} | \hspace{4em} | \hspace{4em} | \hspace{2em} \chord*{\underline{62 64}} |\\
|\chord*{A} \hspace{3.25em} | \hspace{4em} | \hspace{4em} | \hspace{2em} \chord*{\underline{64 62}} | \chord*{E}$\downarrow$\hspace{0.5em}$\downarrow$\hspace{0.5em}$\downarrow$
\end{verse}

\begin{verse}{solo}
Canta una copla la tuna. (¡OLE!)
\end{verse}
\clearpage
\begin{verse}{Refrein}
\chord{A}\hspace{1em}La copla del ronda\chord{E}dor.\\
Canta una copla la \chord{A}tuna,\\
para que salgas mo\chord{E}rena\\
a ver \chord{D}a tu trova\chord{A}dor.\\
para que salgas mo\chord{E}rena (*¡que buena!)\hspace{2em} * alleen laatste keer\\
a ver \chord{D}a tu trova\chord{A$\downarrow$}dor.\chord{E$\downarrow$}\hspace{1em}\chord{A$\downarrow$}
\end{verse}

\textbf{[Intermezzo]}\\
\textbf{[Solo]}\\
\textbf{[Refrein]}\\
\vspace{10em}
\textbf{Vertaling}\\
Hier is de tuna die met haar plezier,\\
en een liedje door de straten loopt.\\
En met haar gitaren\\
en met haar plezier de mensen blij maakt.\\\vspace{1em}
Het zijn de studenten, jongens die van een geintje houden,\\
die mooi kunnen praten, en een groot hart hebben.\\
En het zijn troubadours,\\
die een beetje liefde brengen\\
in hun tonen en woorden, aan meisjes.\\\vspace{1em}
De tuna zingt een lied. OLE !\\\vspace{1em}
Het lied van de straatmuzikant.\\
De tuna zingt een lied,\\
opdat je naar buiten komt meisje,\\
om naar je troubadour te kijken.
