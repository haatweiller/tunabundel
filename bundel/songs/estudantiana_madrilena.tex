\begin{song}[paso_doble]{Estudiantina Madrileña}
\rhythmandkey{Paso doble}{Em/E}

\begin{instrumental}{Intro}[5]
\measure{}\measure{G} \measure{} \measure{} \measure{D7} \measure{} \measure{} \measure{} \measure{G}
\measure{} \measure{E7} \measure{} \measure{Am} \measure{} \measure{G} \measure{D7} \measure{G$\downarrow$}
\measure{\underline{63 62 63 50 52 53}} \measure{B7} \measure{}
\end{instrumental}

\begin{verse}{Couplet 1}
\chord{B7$\downarrow$}\hspace{2em} Por las calles de Ma\chord{Em}drid \chord{(\underline{60 62 63 52 63 62})}\\
\chord{Em}\hspace{2em} Bajo la luz de la \chord{B7}luna,\\
De Cascorro a Chambe\chord{Am}rí,\\
Pasa rondando la \chord{Em}tuna.\\

Su alegría y buen hu\chord{E7}mor\\
Son en la noche abri\chord{Am}leña\\
Como un requiebro de a\chord{Em}mor (de amor, de a\chord{B7}mor)\\
A la mujer madri\chord{E}leña.
\end{verse}

\begin{verse}{Refrein}
Asóma\chord{E}te, asómate al balcón carita de azu\chord{B7}cena \\
Y así ve\chord{F$\sharp$m}rás que pongo en mi can\chord{B7}ción \\
Suspiros de ver\chord{E}bena\chord{A B7 E} \\
Adórnate ciñendote un man\chord{E7}tón de la china, la \chord{A}china \\
Asóma\chord{Am}te, asómate al bal\chord{B7}cón a ver la estudian\chord{E}tina 
\end{verse}

\begin{verse}{Brug (solo)}
Clave\chord{G$\sharp\downarrow$}litos rebo\chord{$\sharp\downarrow$}nitos del jar\chord{A$\downarrow$}dín de \chord{A$\downarrow$}mi Ma\chord{G$\sharp\downarrow$}drid \\
Madri\chord{B$\downarrow$}leña no nos plantes \\
Porque \chord{C$\downarrow$}somos estudi\chord{B$\downarrow$}antes \\
Y can\chord{C$\downarrow$}tamos para \chord{B$\downarrow$}tí
\end{verse}
\repetition{Refrein}
\repetition{Intro}
\end{song}
