\begin{song}[paso_doble]{Estudiantina Portugesa}
\rhythmandkey{Paso doble}{C/A}

\begin{instrumental}{Intro}
\measure{A} \measure{} \measure{} \measure{} \measure{} \measure{} \measure{} \measure{}
\measure{} \measure{} \measure{} \measure{} \measure{E7} \measure{} \measure{} \measure{}
\measure{} \measure{} \measure{} \measure{} \measure{} \measure{} \measure{} \measure{}
\measure{} \measure{} \measure{} \measure{} \measure{A} \measure*{A$\downarrow$}
\end{instrumental}

\begin{verse}{Couplet 1}
    Somos cantores de la tierra Lusitana\\
    Traemos canciones de los aires y del mar\\
    Vamos llenando los balcones ventanas\\
    De melod\'ias del antiguo Portugal
\end{verse}

\begin{verse}{Couplet 2}
    Oport riega en vino rojo sus laderas\\
    De flores rojas va cubierto el litoral\\
    Verde es el Tajo, verde son sus dos riberas\\
    Los dos colores de la ense\~na nacional
\end{verse}

\begin{verse}{Refrein}
    ¿Por qu\'e tu tierra toda es un encanto?\\
    ¿Por qu\'e? ¿Por qu\'e se maravilla qui\'en te ve?\\
    Ay Portugal ¿Por qu\'e te quiero tanto?\\
    ¿Por qu\'e? ¿Por qu\'e te envidian todos, ay, por qu\'e?\\
    \\
    Ser\'a que tus mujeres son hermosas\\
    Ser\'a ser\'a que el vino alegra el coraz\'on\\
    Ser\'a que huelen bien tus lindas rosas\\
    Ser\'a ser\'a qie est\'as ba\~nada por el sol
\end{verse}

\repetition{Couplet 1 Instrumentaal}
\repetition{Couplet 2 Instrumentaal}
\repetition{Refrein}
\begin{instrumental}{Outro}
    \measure{} \measure{Dm} \measure{} \measure{G} \measure{} \measure{A} \measure{} \measure{A$\downarrow$}
\end{instrumental}
\end{song}
