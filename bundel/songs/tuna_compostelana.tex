\begin{song}[paso_doble]{Tuna de Compostalana}
\rhythmandkey{Paso Doble}{D}
\small
\begin{instrumental}{Intro}
\measure{D$\downarrow$;;$\circ$;$\circ$}\measure{D$\downarrow$;;$\circ$;$\circ$}\measure{D$\downarrow$;;$\circ$;$\circ$}\measure*{D$\downarrow$;;$\circ$;$\circ$}[:]
\measure{G$\downarrow$;;$\circ$;$\circ$}\measure{D$\downarrow$;;$\circ$;$\circ$}\measure{A7$\downarrow$;;$\circ$;$\circ$}\measure{D$\downarrow$;;$\circ$;$\circ$}
\measure{Gm$\downarrow$;;$\circ$;$\circ$}\measure{Dm$\downarrow$;;$\circ$;$\circ$}\measure{A7$\downarrow$;;$\circ$;$\circ$}\measure{Dm$\downarrow$}
\end{instrumental}

\begin{verse}{Couplet 1}
\chord{50 52 53}\hspace{4em}\chord{Dm}Pasa la tuna en Santi\chord{Gm}ago\\
cantado muy \chord{Dm}quedo ro\chord{A7}mances de a\chord{Dm$\downarrow$}mor.\\
\chord{50 52 53}\hspace{4em}\chord{Dm}Luego la noche sus \chord{Gm}ecos\\
los cuela de \chord{C}ronda por todo bal\chord{F}cón.\\
Y allá en el \chord{C}templo de Apóstol \chord{F}Santo,\\
una niña \chord{Gm}llora ante su pa\chord{Dm}trón,\\
\begin{sidenote}{2x}
porque la \chord{Gm}capa del tuno que a\chord{Dm}dora\\
no lleva la \chord{A7}cinta que ella le bor\chord{Dm}dó. \chord{D7\textnormal{(1e keer)}}
\end{sidenote}
\end{verse}

\begin{verse}{Refrein}
Cuando la \chord{D}tuna te dé serenata,\\
no te ena\chord{A7}mores, Composte\chord{D}lana.\\
Que cada cinta que adorna mi capa,\chord{}\\
es un tro\chord{A7} de mi cora\chord{D}zón.\hspace{1em}\chord{D7}\\
Ay \chord{G}tray-lay-lay-lay-lay-lay-\chord{D}lay.\\
No te ena\chord{A7}mores, Composte\chord{D}lana. \chord{D7}\\
Y \chord{G}deja la tuna pa\chord{D}sar,\\
con su \chord{A7}tra-la-la-la-la\chord{D$\downarrow$}.\\
\end{verse}

\begin{verse}{Couplet 2}
\chord{50 52 53}\hspace{4em}\chord{Dm}Hoy va la tuna de \chord{Gm}gala\\
cantando y to\chord{Dm}cando la \chord{A7}marcha nup\chord{Dm$\downarrow$}cial.\\
\chord{50 52 53}\hspace{4em}\chord{Dm}Suenan campanas de \chord{Gm}gloria\\
que dejan de\chord{C}sierta la universi\chord{F}dad.\\
Y allá en el \chord{C}templo del Apósto \chord{F}Santo,\\
con el estudi\chord{Gm}ante hoy se va a ca\chord{Dm}sar,\\
\begin{sidenote}{2x}
la galle\chord{Gm}guiña, melosa y ce\chord{Dm}losa,\\
que oyendo esta \chord{A7}copla, ya no llora\chord{Dm}rá. \chord{D7\textnormal{(1e keer)}}
\end{sidenote}
\end{verse}
\repetition{Refrein}
\normalsize
\end{song}

\clearpage
\begin{translation}
De tuna komt langs in Santiago\\
heel zacht liefdesliedjes zingend.\\
De nacht laat ze dan tijdens de ronda\\
op elk balkon weergalmen.\\
En ginds in de tempel van de heilige apostel,\\
weent een meisje om haar beschermer,\\
omdat de capa van de tuno die zij aanbidt,\\
niet het lint draagt dat zij voor hem borduurde.\vspace{\wlskip}

Wanneer de tuna je een serenade geeft,\\
word dan niet verliefd, Compostelana.\\
Want elke lintje dat mijn capa siert,\\
is een stukje van mijn hart.\\
Ay tra-la-la-la-la-la-la\\
Word niet verliefd, Compostelana.\\
En laat de tuna voorbijgaan\\
met zijn tra-la-la-la-la.\vspace{\wlskip}

Nu gaat de tuna in galakostuum\\
zingend en spelend de bruiloftsmars.\\
De klokken luiden van vreugde\\
die de universiteit verlaten doen zijn.\\
En ginds in de tempel van de heilige apostel\\
gaat nu met een student trouwen\\
de lieflijke, jaloerse gallicische,\\
die na het horen van deze copla niet meer zal wenen.
\end{translation}
