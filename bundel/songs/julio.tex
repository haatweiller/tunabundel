\begin{song}[paso_doble]{La Morena De Mi Copla / Julio}
\rhythmandkey{Paso doble}{Am/A}

\begin{instrumental}{Intro}
\measure[:]{E*}\measure{;;;F$\downarrow$}\measure{E*}\measure*{}[:]
\measure{;64;50;52}\measure{C}\measure{;;52;50}\measure{G;;63;62}\measure{F}\measure{}\measure{E$\downarrow$;F$\downarrow$}\measure{E$\downarrow$}
\end{instrumental}

\begin{verse}{Couplet 1}
\chord{E}Julio Romero de Torres\\
Pin\chord{F}tó a la mujer mo\chord{E}rena,\\
Con los ojos de misterio\chord{}\\
Y el \chord{F}alma \chord{G}llena de \chord{C}pena.\chord{52 50}\\
P\chord{G}uso en sus brazos\phantom{x}\chord{50}\phantom{xxx}\chord{52} de b\chord{C}ronce\\
La guita\_\_\_\_\_rra \chord{52 50}canta\_\chord{E}ora.\\
\chord{F}En su bordón hay su\chord{E}spiros\hspace{4em}
\chord{F}Y en su caja una do\chord{E}lora.\\
\end{verse}

\begin{instrumental}{Brug}
\measure{Dm*}\measure{}\measure{B7*}\measure{}\measure{F*}\measure{}\measure{E$\downarrow$; F$\downarrow$}\measure{E$\downarrow$}
\end{instrumental}

\begin{verse}{Refrein}
\chord{60 62 64}\hspace{6ex}Mor\chord{A}ena, la de los \chord{Edim}rojos cla\chord{E}veles\\
La de la reja flo\chord{A}ría.\\
La \chord{50\phantom{x}64\phantom{x}62\phantom{x}60\phantom{x}\ViPa}reina de las\hspace{\wlskip} mu\chord{E}jeres. \chord{\underline{42 41 40}}\\
Mo\chord{D}rena, \chord{\phantom{xxx}40 54 52 50\phantom{xx} E}\small{(la shubi\_du\_bi\_du)} la del bordao man\chord{A}tón, \small{(ding dong ding dong)}\\
La de la a\chord{A7}legre gui\chord{D}tarra, \chord{Dm}\\
\chord{A}La del cla\chord{E}vel espa\chord{A$\downarrow$}ñol\chord{E$\downarrow$ A$\downarrow$}.
\end{verse}

\repetition{Intro}

\begin{verse}{Couplet 2}
\chord{E}Como escapada del cuadro,\\
En \chord{F}el sentir de la \chord{E}copla,\\
Toda España la venera,\chord{}\\
Y \chord{F}toda Es\chord{G}paña la \chord{C}llora.\chord{52 50}\\
\chord{G}Trenza con su \chord{50}ta\_\chord{52}co\_\chord{C}neo, (¡soy de Holanda! \small{(ding dong)}, 2x)
\end{verse}

\begin{verse}{solo}
la seguidi\_\_\_\_\_\_\_\_\chord{52}lla \chord{50}de Es\_\chord{E}paña
\end{verse}

\begin{verse}{}
\chord{F}En su danzar es mo\chord{E}runa,\hspace{4em}
\chord{F}En la venta de Eri\chord{E}taña.

\end{verse}
\repetition{Brug}
\repetition{Refrein}
\begin{center}
\large{Freeze}
\end{center}
\end{song}

\begin{translation}
Julio Romero de Torres\\
schilderde de donkerbruine vrouw,\\
met mysterieuze ogen\\
en de ziel vol smart.\\
In haar bronskleurige armen legde hij\\
de gitaar van een liedjeszanger;\\
op haar lippen een zucht\\
en op haar gelaat een droevige blik.\vspace{\wlskip}

Morena, die van de rode anjelieren,\\
die van het traliewerk vol bloemen.\\
De koningin van de vrouwen.\\
Morena, die van de geborduurde omslagdoek,\\
die van de vrolijke gitaar,\\
die van de Spaanse anjelier.\vspace{\wlskip}

Als ontsnapt uit een schilderijlijst\\
bij het horen van het lied.\\
Heel Spanje vereert haar\\
en heel Spanje beweent haar.\\
Zij stapt statig met gestamp van haar hakken\\
de volksdans van Spanje.\\
In haar zingen is ze een Moorse,\\
in de herberg van Eritaña.
\end{translation}
