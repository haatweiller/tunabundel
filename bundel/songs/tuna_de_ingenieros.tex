\begin{song}[paso_doble]{Tuna de Ingenieros}
\rhythmandkey{Paso doble}{A}

\begin{verse}{Intro}
%\begin{instrumental}{Intro}
\textbf{TODO:} Dit schrijven
%\end{instrumental}
\end{verse}

\begin{verse}{Couplet}
Va cay\chord{A}endo ya la noche en la ciudad\\
y se acerca la alegre estudiant\chord{E}ina\\
al pandero se oye marcando el compás\chord{}\\
a los tunos de la viela y la turb\chord{A}ina.\\
Las mujeres quedan todas temblorosas\chord{}\\
al sentir debajo s\chord{A7}u balcón cant\chord{D}amos.\\
Y celosas, recelosas, tembler\chord{A}osas\\
se les h\chord{F\#m}ace el culo gase\chord{Bm}osa.\\
Ante aq\chord{E}uesta donosura que gast\chord{A}amos.
\end{verse}

\begin{verse}{Brug}
\phantom{xxxxx}\chord{\underline{60 62 64}}L\chord{A}a de Ingenieros llegó.\\
Cantando va y su alegria a la ciudad dej\chord{E}ó.\\
Y un tuno con aire vacilón\chord{}\\
te alegrará el coraz\chord{A}ón, mujer.\chord{\underline{60 62 64}}
\end{verse}

\begin{verse}{Refrein}
D\chord{A}éjate amar por él\\
porque podrás enloquec\chord{A7}er de am\chord{D}or.\\
Y sentir la vida llena de aleg\chord{A}ria,\chord{F\#m}\\
de aleg\chord{Bm}ria y b\chord{E}uen hum\chord{A$\downarrow$ E$\downarrow$ A$\downarrow$}or.
\end{verse}

\textbf {Brug} (Instrumentaal)\\
\textbf {Refrein}
\end{song}
