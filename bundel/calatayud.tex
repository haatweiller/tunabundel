\begin{song}{Calatayud}
\rhythmandkey{Paso Doble}{A}
\begin{instrumental}{Intro}
\measure{}\measure{A}\measure{}\measure{}\measure{}\measure*{}\\
\measure{E}\measure{}\measure{}\measure{}\measure{}\measure*{}\\
\measure{A}\measure{}\measure{}\measure{}\measure{}\measure*{}\\
\measure{E}\measure{}\measure{}\measure{}\measure{}\measure*{}\chord*{A$\downarrow$}\hspace{1.5em}\chord*{E$\downarrow$}\chord*{A$\downarrow$}\\
\end{instrumental}

\begin{verse}{Couplet 1}
Por ser am\chord{A}iga de hacer fa\chord{E}vores.\\
Por ser alegre su juven\chord{A$\downarrow$}tud.\hspace{0.5em}\chord{E$\downarrow$}\hspace{0.5em}\chord{A$\downarrow$}\\
En \chord{Am}coplas se vió la Do\chord{E}lores,\\
la flor de Calata\chord{Am (slagje)}yud.\\
Una \chord{G}jotica recorrió Es\chord{C}paña,\\
pregón de in\chord{F}famia de una mu\chord{E}jer.\\
Y el buen nombre\hspace{0.5em}\chord{E$\downarrow\downarrow$}\hspace{0.5em} de aquella maña\chord{Am$\downarrow\downarrow$}\\
\chord{*Dm}yo tengo \chord{F}que defen\chord{E}der.\hspace{3em}(* roffelen)\\
\end{verse}

\begin{verse}{Refrein}
Si vas a Calata\chord{A}yud, si vas a Calata\chord{E}yud.\\
Pregunta por la Do\chord{A}lores, una copla la ma\chord{E}tó,\\
de ver\chord{D}güenza y sin sa\chord{A}bores.\\
\chord{}Ves que te lo digo yo,\\
el hijo de la Dol\chord{A}ores.\chord{} \hspace{1.5em}\chord{A$\downarrow$}\hspace{1.5em}\chord{E$\downarrow$}\hspace{1.5em}\chord{A$\downarrow$}\\
\end{verse}

\begin{verse}{Couplet 2}
Dice la \chord{A}gente de mala \chord{E}lengua,\\
que por tu calle me ve ron\chord{A$\downarrow$}dar.\chord{E$\downarrow$}\hspace{1.5em}\chord{A$\downarrow$}\\
"¿ Tú \chord{Am}sabes su madre quien \chord{E}era ?\\
Dolores la del can\chord{Am}tar ."\\
Yo la que\chord{G}ría con amor \chord{C}bueno,\\
más la ca\chord{F}lumnia la averg\chord{E}onzó.\\
Y no supo\hspace{0.5em}\chord{E$\downarrow$ $\downarrow$}\hspace{0.5em} limpiar el cieno,\chord{Am $\downarrow$ $\downarrow$}\\
\chord{*Dm}que la mal\chord{F}dad le arro\chord{E}jó.\\
\end{verse}
\textbf{[Refrein]}
\end{song}
