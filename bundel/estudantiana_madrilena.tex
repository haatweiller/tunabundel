\begin{song}[paso_doble]{Estudiantina Madrileña}
\rhythmandkey{Paso doble}{Em/E}

\begin{instrumental}{Intro}
Opmaat bandurria
\measure{G} \measure{} \measure{} \measure{D7} \measure{} \measure{} \measure{} \measure*{G}
\measure{} \measure{E7} \measure{} \measure{Am} \measure{} \measure{G} \measure{D7} \measure*{G$\downarrow$}
\measure{\underline{63 62 63 50 52 53}} \measure{B7} \measure*{}
\end{instrumental}

\begin{verse}{Couplet 1}
\chord{B7}\hspace{2em} Por las calles de Madrid \chord{\underline{60 62 63 52 63 62}}\\
\chord{Em}\hspace{2em} Bajo la luz de la \chord{B7}luna,\\
De Cascorro a Chambe\chord{Am}rí,\\
Pasa rondando la \chord{Em}tuna.\\
\end{verse}

\begin{verse}{Couplet 2}
Su alegría y buen hu\chord{E7}mor\\
Son en la noche abri\chord{Am}leña\\
Como un requiebro de a\chord{Em}mor (de amor, de amor)\\
A la mujer madri\chord{E}leña.\\
\end{verse}

\begin{verse}{Refrein 1}
Y asómate, asómate al balcón
Carita de azucena,
Y así verás que pongo en mi canción
Suspiros de verbena.
Adórnate ciñéndote un mantón
De la China, la China,
Asómate, asómate al balcón

A ver la estudiantina.
\end{verse}

\textbf {Intro} \\
\textbf {Coupletten}\\
\vspace{1.25ex}
\textbf{Einde} of: \textbf{ No me caso}\\
\begin{verse}{Einde}
para lle\chord{D}varla a bai\chord{A7}lar al cuar\chord{D}tel.\\
para lle\chord{D}varla a bai\chord{A7}lar al cuar\chord{D}tel.\\
\end{verse}
\end{song}
