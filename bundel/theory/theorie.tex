\section{Theorie}
\subsection*{Inleiding}
Uit oude tuna bundel geciteerd. Deze was geschreven door Michiel, Peter en Robert 8 oktober 2004.
\subsection*{Toonladders}
In het westerse toonsysteem wordt een octaaf opgedeeld in 12 noten. Zelden worden al deze twaalf noten gebruikt in een muziekstuk: er wordt een selectie gemaakt. Soms worden voor verschillende delen van een stuk verschillende selecties gebruikt, maar op ieder moment is er slechts één selectie van noten die gebruikt wordt. Zo’n selectie van noten heet een toonladder.

De meest bekende toonladder is de \chord*{C}-majeur toonladder. In het volgende schema staat weergegeven hoe deze toonladder is opgebouwd:

\begin{tabular}{*{13}{c}}
 1 &    & 2 & & 3 & & 4 & & 5 & & 6 & & 7\\
 \chord*{C} &    & \chord*{D} & & \chord*{E} & & \chord*{F} & & \chord*{G} & & \chord*{A} & & \chord*{B}\\
   & +2 & & +2 & & +1 & & +2 & & +2 & & +1
\end{tabular}

De plaats van een bepaalde noot in de toonladder wordt aangegeven met een cijfer. Zo is de \chord*{D} bijvoorbeeld de tweede noot in de \chord*{C}-majeur toonladder en wordt derhalve aangegeven met een 2. Evenzo \chord*{E} = 3, \chord*{F} = 4, etc.

Zoals in het schema staat aangegeven zit er niet tussen elke twee opeenvolgende noten een gelijke afstand. Zo zit bijvoorbeeld tussen de \chord*{C} en de \chord*{D} op de piano en zwarte toets en tussen de \chord*{E} en de \chord*{F} niet. Op de gitaar ligt de \chord*{D} twee frets hoger dan de \chord*{C} en de \chord*{F} maar één hoger dan de \chord*{E}.

\subsubsection*{Intervallen}
Als je twee willekeurige noten uit de \chord*{C}-majeur toonladder samen of na elkaar speelt, speel je een interval. Een interval is een selectie van twee noten. De verschillende intervallen die je in de \chord*{C}-majeur toonladder vanaf de \chord*{C} kan maken staan in het volgende overzicht weergegeven:

\begin{tabular}{l l rl}
\chord*{C}-\chord*{C} &	prime   & 0 & frets\\
\chord*{C}-\chord*{D} & secunde & 2 & frets\\
\chord*{C}-\chord*{E} & terts	& 4 & frets\\
\chord*{C}-\chord*{F} & kwart	& 5 & frets\\
\chord*{C}-\chord*{G} & kwint	& 7 & frets\\
\chord*{C}-\chord*{A} & sext	& 9 & frets\\
\chord*{C}-\chord*{B} & sept	& 11 &frets\\
\chord*{C}-\chord*{C’} & oktaaf & 12 & frets
\end{tabular}
\clearpage
Alle mogelijke intervallen die je in het westerse toonsysteem binnen een octaaf kunt maken staan in onderstaande tabel weergegeven:\\
\begin{tabular}{l l r l}
\chord*{C}-\chord*{C}			&	prime				&0  &frets\\
\chord*{C}-\chord*{D$\flat$}	&	kleine secunde		&1  &frets\\
\chord*{C}-\chord*{D}			&	grote secunde		&2  &frets\\
\chord*{C}-\chord*{D$\sharp$}	&	overmatige secunde	&3  &frets\\
\chord*{C}-\chord*{E$\flat$} 	&	kleine terts		&3  &frets\\
\chord*{C}-\chord*{E}			&	grote terts			&4  &frets\\
\chord*{C}-\chord*{F$\flat$}	&	verminderde kwart	&4  &frets\\
\chord*{C}-\chord*{F}			&	reine kwart			&5  &frets\\
\chord*{C}-\chord*{F$\sharp$} 	&	overmatige kwart	&6  &frets\\
\chord*{C}-\chord*{G$\flat$} 	&	verminderde kwint	&6  &frets\\
\chord*{C}-\chord*{G}			&	reine kwint			&7  &frets\\
\chord*{C}-\chord*{G$\sharp$}	&	overmatige kwint	&8  &frets\\
\chord*{C}-\chord*{A$\flat$} 	&	kleine sext			&8  &frets\\
\chord*{C}-\chord*{A}			&	grote sext			&9  &frets\\
\chord*{C}-\chord*{B$\flat\flat$} & verminderde sept	&9  &frets\\
\chord*{C}-\chord*{B$\flat$}	&	kleine sept			&10 &frets\\
\chord*{C}-\chord*{B}			&	grote sept			&11 &frets\\
\chord*{C}-\chord*{C’}			&	oktaaf				&12 &frets
\end{tabular}

Je kan een toonladder zien als een grondtoon waarop een verzameling intervallen is gedefinieerd.

\subsubsection*{Majeur toonladders}
De majeur toonladder is opgebouwd uit de volgende intervallen: een grote secunde, een grote terts, een reine kwart, een reine kwint, een grote sext en een grote sept. 

Op deze manier kan je op iedere willekeurige toon een majeur toonladder bouwen. Zo wordt bijv \chord*{G}-majeur:

\chord*{G} \chord*{A} \chord*{B} \chord*{C} \chord*{D} \chord*{E} \chord*{F$\sharp$}

De belangrijkste majeur toonladders binnen tuna muziek zijn:

\begin{tabular}{l l l l l l l l}
           &1  &2  &3  &4  &5  &6  &7\\
A-majeur	&A  &B  &C$\sharp$ &D  &E  &F$\sharp$ &G$\sharp$\\
D-majeur	&D  &E  &F$\sharp$ &G  &A  &B  &C$\sharp$\\
E-majeur	&E  &F$\sharp$ &G$\sharp$ &A  &B  &C$\sharp$ &D$\sharp$
\end{tabular}

\subsubsection*{Mineur toonladders}
De mineur toonladder is opgebouwd uit de volgende intervallen: een kleine secunde, een kleine terts, een reine kwart, een reine kwint, een kleine sext en een kleine sept.

Zo wordt \chord*{A}-mineur:

\chord*{A} \chord*{B} \chord*{C} \chord*{D} \chord*{E} \chord*{F} \chord*{G}

Zoals je ziet is dit dezelfde selectie van noten als de \chord*{C}-majeur toonladder alleen begin je nu op \chord*{A} in plaats van op \chord*{C}. Dit geldt voor iedere majeur toonladder: als je op de zesde noot begint in plaats van op de eerste, krijg je een mineur toonladder. Deze mineur toonladder noem je de relatieve mineur toonladder van de bij behorende majeur toonladder. Zo is \chord*{A}-mineur de relatieve mineur toonladder van \chord*{C}-majeur en \chord*{E}-mineur de relatieve mineur toonladder van \chord*{G}-majeur.

\begin{tabular}{l l}
Majeur	& Relatief mineur\\\hline
\chord*{A}&	\chord*{F$\sharp$}\\
\chord*{D}&	\chord*{B}\\
\chord*{E}& \chord*{C}
\end{tabular}

\begin{tabular}{l l}
Mineur	& Relatief majeur\\\hline
\chord*{A}& \chord*{C}\\
\chord*{D}&	\chord*{F}\\
\chord*{E}&	\chord*{G}
\end{tabular}

\hspace{50pt}\textbf{Harmonisch mineur}

\scriptsize
\begin{tabular}{*{12}{p{0.2em} p{0.1em} } p{1em}}
1 &    & 2 &    & 3 &    & 4 &    & 5 &    & 6 &    & 7 & & 1\\
  &  +2 &   &  +2 &   &  +1 &   &  +2 &   &  +2 &   &  +2 & &  +1\\
C &    & D &    & E &    & F &    & G &    & A &    & B & & C\\
  &    &   &    &   &    &   &    &   &    & A &    & B & & C & & D & & E & & F & & G & & A\\
  &    &   &    &   &    &   &    &   &    &  &  +2 & &  +1 & &  +2 & &  +2 & &  +1 & &  +2 & &  +2\\
  &    &   &    &   &    &   &    &   &    & 1 & & 2 & & 3 & & 4 & & 5 & & 6 & & 7 & & 1
\end{tabular}

\normalsize
\hspace{190pt}\textbf{Frygisch mineur}

\subsection*{Akkoorden}
Als je op een instrument meerder noten tegelijk speelt, speel je een akkoord. Het basis element van een akkoord is de zogenaamde drieklank.

\subsubsection*{Drieklanken}
Als je in een toonladder op een noot twee keer een terts omhoog gaat (bijv \chord*{C} \chord*{E} \chord*{G}, of \chord*{G} \chord*{B} \chord*{D} in \chord*{C}-majeur) en deze 3 noten samen speelt heb je een drieklank. Speel je deze na elkaar heet het een gebroken drieklank. Een drieklank kan, afhankelijk van of je grote of kleine tertsen gebruikt, verschillende karakters hebben. De 2 belangrijkste zijn majeur (grote terts+kleine terts) en mineur (kleine terts+grote terts).
Als je in \chord*{C}-majeur een drieklank

Als je op de grondtoon (eerste toon) van een majeur toonladder een drieklank bouwt krijg je een majeur drieklank, als je dit in een mineur toonladder doet krijg je een mineur drieklank.

\subsubsection*{Trappen}
In de majeur en mineur toonladders zitten zeven noten. Elk van deze noten kan je gebruiken als grontdtoon voor een drieklank. In iedere toonladder zijn dus ook maar zeven drieklanken mogelijk. Deze worden trappen genoemd:

\begin{tabular}{| l | l l l | l l l |}
\hline
Trap & \multicolumn{3}{ l |}{Majeur toonladder} & \multicolumn{3}{ l |}{Mineur toonladder}\\
& & & & & & \\
& maj/min & C-majeur & not.	& maj/min & A-mineur & not. \\
\hline
Eerste trap	& maj	&	C	&	I	& min	&	Am	&	i	\\
Tweede trap	& min	&	Dm	&	ii	& -5	&	B-5	&	ii	\\
Derde trap	& min	&	Em	&	iii	& maj	&	C	&	III	\\
Vierde trap	& maj	&	F	&	IV	& min	&	Dm	&	iv	\\
Vijfde trap	& maj	&	G	&	V	& min	&	Em	&	v	\\
Zesde trap	& min	&	Am	&	vi	& maj	&	F	&	VI	\\
Zevende trap & -5	&	B-5	&	vii	& maj	&	G	&	VII	\\
\hline
\end{tabular}

\chord*{B-5} is het akkoord dat je krijgt als je in de \chord*{C}-majeur of \chord*{A}-mineur toonladder \chord*{B} gebruikt als grondtoon voor een drieklank. Deze bestaat uit de noten \chord*{B}, \chord*{D} en \chord*{F}. Het interval \chord*{B}-\chord*{F} is een verminderde kwint (2 kleine tertsen op elkaar gestapeld, 6 frets). Deze komt alleen voor bij de zevende trap in een majeur toonladder en de tweede trap in een mineur toonladder. De rest van de trappen bestaan uit een grote/kleine terts (majeur/mineur) en een reine kwint

\subsubsection*{Van drieklank naar akkoord}
Als je op de gitaar het E mineur akkoord aanslaat, speel je geen drie, maar zes noten. De noten die je speelt zijn van laag naar hoog:

\chord*{E} \chord*{B} \chord*{E} \chord*{G} \chord*{B} \chord*{E}

De \chord*{E} wordt in totaal drie keer gespeelt en de \chord*{B} twee keer. De basis voor dit akkoord is de \chord*{E} mineur drieklank \chord*{E} \chord*{G} \chord*{B}. Er zijn twee trucjes om van een drieklank een akkoord te maken:

\paragraph*{Verdubbeling:}
Iedere noot uit de drieklank kan je verdubbelen door de zelfde noot een octaaf hoger of lager toe te voegen. Je kan bijvoorbeeld de \chord*{G} in het \chord*{Em} akkoord verdubbelen door 0-2-2-0-0-3 te spelen. Nu speel je twee keer \chord*{E}, twee keer \chord*{B} en twee keer \chord*{G}.

\paragraph*{Inversie:}
De \chord*{E} mineur drieklank \chord*{E} \chord*{G} \chord*{B} kan je ook spelen als \chord*{G} \chord*{B} \chord*{E} (1e inversie) of \chord*{B} \chord*{E} \chord*{G} (2e inversie). Over het algemeen is het het mooist als de laagste toon van je akkoord de grondtoon van het akkoord is. Zo kan je het \chord*{E} mineur akkoord ook spelen als 3-2-2-0-0-0. Nu is de \chord*{G} de laagste toon in je akkoord in plaats van de grondtoon \chord*{E}. Het \chord*{E} mineur akkoord op de gebruikelijke manier gespeeld (E als grondtoon) klinkt echter in 99\% van de gevallen beter.

\paragraph*{Samenvattend:} iedere combinatie van de drie noten die een drieklank vormen is een akkoord met dezelfde naam als de drieklank.

\subsection*{Akkoord progressies}
Als je een nummer speelt komt het zelden voor dat je slechts 1 akkoord gebruikt. Vrijwel altijd speel je combinatie of een reeks van akkoorden. Zo’n reeks heet een akkoord progressie. Een akkoordprogressie is het middel bij uitstek om een spanningsboog te creeren (tension/release). De verschillende trappen van een toonladder hebben allemaal hun eigen functie binnen deze spanningsboog. Het komt zelden voor (behalve in klassiek) dat alle zeven trappen van een toonladder worden gebruikt. Meestal (zeker in tuna nummers) worden maar drie akkoorden/trappen uit een toonladder gebruikt. Deze drie trappen zullen hieronder worden toegelicht.

\subsubsection*{De eerste trap I (we zijn thuis)}
De eerste trap legt de toonsoort vast waarin het stuk gespeeld wordt. De meeste muziekstukken beginnen op de eerste trap en eindigen op de eerste trap. Als een muziekstuk op een andere trap dan de eerste begint geeft dit als het ware het gevoel alsof je op een rijdende trein springt: het muziekstuk is al bezig en je stapt er midden in (bijv. Calles sin Rumbo). Als je op een andere trap eindigd dan de eerste krijg je een open einde, je oren verwachten nog de eerste trap. Je kan een happy ending krijgen door op de eerste trap in majeur te eindigen (ookal is het hele stuk in mineur). Eindigen op de eerste trap in mineur laat je met een melancholiek namijmerend gevoel over.
In een akkoordprogressie geeft de eerste trap het gevoel van we zijn thuis. Zolang de eerste trap wordt gespeeld is er geen spanning aanwezig. Als een andere trap dan de eerste wordt gevolgd door de eerste trap is de aanwezige spanning verdwenen (release).
 
\subsubsection*{De vierde trap IV (we gaan op reis)}

\subsubsection*{De vijfde trap V (we hebben heimwee)}
De vijfde trap
\subsubsection*{Het dominant septiem akkoord (we willen nu naar huis!)}

\begin{tabular}{l l l l l}
		 & I & IV & V & V7\\
A-majeur &	A &	D &	E & E7\\
D-majeur &	D &	G &	A & A7\\
E-majeur &	E &	A &	B & B7\\
A-mineur &	Am & Dm & E & E7\\
D-mineur &	Dm & Gm & A	& A7\\
E-mineur &	Em & Am & B	& B7
\end{tabular}

\subsection*{Vraag Antwoord}
\subsubsection*{Syntax}
\begin{tabular}{*{10}{| p{2.5em}@{}}}
\multicolumn{4}{|l|}{Statisch} & \multicolumn{4}{l|}{Dynamisch} & \multicolumn{2}{l|}{Cadens}\\\hline
\vspace{1pt}I & V & & I\hfill :&I & IV & I & V7 & I \hfill V7 & I\\
A & E7 & & A\hfill :&A & D & A & E7\hfill :& A \hfill E7 & A	
\end{tabular}

\clearpage
\subsection*{Tuna Progressies}

Andalusische cadens
In spaanse muziek 

Voorbeelden:\vspace{-20pt}
\begin{itemize}
\item Alma Corazon y vida
\item Espanola
\item Granada (intro)
\end{itemize}

In aangepaste vorm\vspace{-20pt}
\begin{itemize}
\item Julio
\item Torrero 
\end{itemize}

I V

I IV I V

I IV V

\clearpage
\subsection*{De Tuna tricks}
\subsubsection*{Relatief majeur/mineur}

In veel tuna nummers (vooral de zuidamerikaanse) wordt een majeurtoonladder gebruikt in combinatie zijn relatieve mineur toonladder.
De duidelijkste voorbeelden hiervan zijn de zuidamerikaanse nummers die in de zogenaamde Huayño stijl zijn. Dit zijn Hoy Estoy Aqui, Naranjitai, Recuerdos Bolivianas en Estampa Cumanesa. Het beste voorbeeld is Naranjitai. De akkoorden hiervan zijn:
\vspace{-24pt}
\begin{instrumental}{}[6]
\measure{Gmajeur} \measure*{Emineur}
\measure{<\dashfill >} \measure*{<\dashfill >}
\measure{D7;G} \measure*{B7;Em}
\measure{V7;I} \measure*{V7;i}
\end{instrumental}

Deze progressie (V I in majeur, V i in rel. mineur) is de backbone van veel muziek uit het Andes gebied (indianen met panfluiten). 
In iets andere vorm komen we dit zelfde principe tegen in Hoy Estoy Aqui. Nu wordt ook het IV akkoord uit de majeur toonladder gebruikt:

Intro:

\begin{tabular}{| *{8}{p{4em}@{}} |}
\multicolumn{6}{c}{Cmajeur} & \multicolumn{2}{c}{Amineur}\\
\multicolumn{6}{|p{24em}@{}|}{<\dashfill >} & \multicolumn{2}{p{8em}@{}|}{<\dashfill >}\\
\multicolumn{1}{|p{4em}@{}}{F} & G & C & \multicolumn{1}{|p{4em}@{}}{: C} & \multicolumn{1}{p{4em}@{}|}{G\hfill :} & C & E7 & Am\\
\multicolumn{1}{|p{4em}@{}}{IV} & V & I & \multicolumn{1}{|p{4em}@{}}{\hspace{7pt}I} & \multicolumn{1}{p{4em}@{}|}{V} & I & V & i
\end{tabular}

Refrein (Couplet):
\vspace{-24pt}
\begin{instrumental}{}[6]
\measure{Cmajeur} \measure*{Amineur}
\measure{<\dashfill >} \measure*{<\dashfill >}
\measure{F;C} \measure*{E7;Am}
\measure{IV;I} \measure*{V7;i}
\end{instrumental}

In het volgende overzicht staan de eerste, vierde en vijfde trap voor een aantal majeur toonladders en de bijbehorende relatieve mineurtoonladders.

\begin{tabular}{*{7}{l}}
 & \multicolumn{3}{c}{Majeur} & \multicolumn{3}{c}{Relatief mineur}\\
 & I & IV & V7 & i & iv & V7\\
A-majeur & A & D & E7 & F$\sharp$m & Bm & C$\sharp$7\\
D-majeur & D & G & A7 & Bm & Em & F$\sharp$7\\
E-majeur & E & A & B7 & Cm & Fm & G7
\end{tabular}

\begin{tabular}{*{7}{l}}
 & \multicolumn{3}{c}{Mineur} & \multicolumn{3}{c}{Relatief majeur}\\
 & i & iv & V7 & I & IV & V7\\
A-mineur & Am & Dm & E7 & C & F & G\\
D-mineur & Dm & Gm & A7 & F & B$\flat$ & C7\\
E-mineur & Em & Am & B7 & G & C & D7
\end{tabular}

\subsubsection*{Substitutie}
In een aantal traditionele tuna nummers worden de relatieve mineur/majeur toonladders net iets anders gebruikt dan in de zuidamerikaanse nummers. In de zuidamerikaanse nummers lost een majeur progressie op in een mineur cadens. In de traditionele tunanummers worden akkoorden in een majeur (of mineur) vervangen door hun relatieve mineur (of majeur) tegenhanger om in een mineur (of majeur) couplet een majeur (of mineur) stuk te krijgen. Voorbeelden van tuna nummers waarin je dit (in wisselende mate) kunt zien zijn:

Calatayud, Clavelitos, Himno de Ingenieros, Julio, El Parandero, Parate, Piel Canela, Rondalla, El Torero, La Tuna Compostelana, en Venezuela.

Deze substitutie wordt in tuna nummers bijna alleen maar gebruikt in coupletten en als brug. Het refrein blijft in een toonaard (majeur/mineur). Voorbeeld: 

Clavelitos

In de coupletten zijn de eerste vier regels in \chord*{A}mineur en wordt er gewisseld tussen het I en het V akkoord. Na “mucha verguenza ni poca..” kondigt het loopje op de gitaar (en op de bandurria/laud) de overgang naar relatief majeur aan. De volgende twee regels zijn in \chord*{C} en hier wordt ook weer gewisseld tussen het I en het V akkoord. Het couplet eindigt vervolgens weer in \chord*{Am}.

Wat hiermee bereikt wordt is dat in een couplet met droevige/statige/melancholische mineur sfeer enkele regels van hoop worden ingebouwd. Als je kijkt naar de tekst van Clavelitos, kan je zien dat in de regels die in Amineur zijn er wordt gevraagd/gesmacht/gemijmerd en in de regels in majeur wordt in het eerste couplet een belofte gemaakt en in het tweede couplet zag hij de anjer in het haar (thema van het nummer, heet immers clavelitos). 

In Tuna Compostelana komt hetzelfde duidelijk terug. In het eerste couplet wordt naar majeur overgeschakeld op het moment dat de liedjes over het balkon weergalmen. Bij “una nina llora” (een meisje huilt) wordt precies op llora (huilt) weer teruggegaan naar mineur. Het moment van hoop is daar dan ook weer over. (in het tweede couplet vindt de overgang naar mineur trouwens plaats als de student gaat trouwen!).

Ook in Rondalla komt dit principe heel duidelijk terug (zie verder).

\subsubsection*{Parallel majeur/mineur}

In veel tuna nummers zijn de coupletten in een mineur toonsoort en wordt voor het refrein de parallel majeur toonladder gebruikt. De coupletten zijn dan bijvoorbeeld in Am (of Em) en het refrein in A (of E). Nummers waarin dit voorkomt zijn:

Alma Llanera, Calatayud, Calles sin Rumbo, Cintas de mi Capa, Clavelitos, Manolo, Julio, Perfidia, Rondalla, El Torero, Tuna Compostelana, Venezuela.

Een goed voorbeeld van een niet-tuna nummer waar dit ook zeer effectief gebruikt wordt is Het kleine cafe aan de haven.

Wat met deze truc bereikt wordt is dat eerst een droevig verhaal verteld wordt in de coupletten, met een positieve/vrolijke boodschap in het refrein.

In Tuna Compostelana gaat de tekst over een meisje dat droevig achterblijft nadat de tuna is langsgekomen en in het refrein komt de boodschap: wordt niet verliefd en laat die tuna maar lekker flierefluiten. Uiteindelijk komt alles goed en trouwt het meisje met haar tuno en tralala’t de tuna lekker verder (3x tralalala op het eind).

Ook Rondalla is hierin weer een goed voorbeeld. In het mineur gedeelte vertelt de zanger dat hij zijn liefde aan een meisje gaat verklaren, in het relatief majeur gedeelte (y en mi corazon..) vertelt over wat hij voor haar voelt en in het parallel majeur gedeelte (refrein) staat hij onder het balkon het zingt uit volle borst: Open je balkon en ook je hart. In de een na laatste regel van het refrein (y este noche) wordt nog kort van D op Dm overgeschakeld (want het is een stille nacht) om meteen weer terug naar majeur te gaan (maar wel de beste van zijn leven!).

\clearpage
\subsubsection*{De tussendominant truc}
Voorbeelden:

Adelita, No Me Caso, Adios, Alma Corazon y Vida, Aurora, Calles sin Rumbo, Cintas de mi Capa, Clavelitos, Despierta, Estrellita del Sur, Himno de Ingenieros, Manolo, Moliendo Cafe, Julio, Perfidia, Piel Canela, Rondalla, San Cayetano, El Torero, Tuna Compostelana, 
\begin{instrumental}{}
\measure{I} \measure{I7} \measure{IV} \measure{} \measure{I} \measure{V7} \measure{I;V7} \measure{I}
\end{instrumental}

\clearpage
\subsection*{Voorbeeld: San Cayetano}

\chord*{c} = cadans: \chord*{A$\downarrow$} \chord*{E7$\downarrow$} \chord*{A$\downarrow$} (I V I)

\subsubsection*{intro:}
De intro van San Cayetano is een statische progressie die wordt afgesloten met een cadans. De zin I V   I wordt twee keer herhaald en de tweede keer wordt afgesloten met de cadans om het einde van intro te aan te geven: 

\begin{instrumental}{}
\measure{1}	\measure{2}	\measure{3}	\measure{4}	\measure{5}	\measure{6}	\measure{7}	\measure{8}
\measure{A}	\measure{E}	\measure{}	\measure{A}	\measure{A}	\measure{E}	\measure{}	\measure{c}
\measure{I}	\measure{V}	\measure{}	\measure{I}	\measure{I}	\measure{V}	\measure{}	\measure{c}
\end{instrumental}

Een statische progressie als intro zet de toon van het nummer en schept verwachting voor wat komen gaat: er wordt nog niet teveel van de sluier opgelicht. 

\subsubsection*{Coupletten en refrein:}

\begin{instrumental}{}
\measure{9}	\measure{10}	\measure{11}	\measure{12}	\measure{13}	\measure{14}	\measure{15}	\measure{16}
\measure{A}	\measure{}	\measure{}	\measure{E}	\measure{E}	\measure{}	\measure{}	\measure{A}
\measure{I}	\measure{}	\measure{}	\measure{V}	\measure{V}	\measure{}	\measure{}	\measure{I}
\end{instrumental}

\vspace{-10pt}Hay..
\vspace{-10pt}
\begin{instrumental}{}
\measure{17} \measure{18} \measure{19} \measure{20} \measure{21} \measure{22} \measure{23} \measure{24}
\measure{A}	\measure{}	\measure{A7}	\measure{D}	\measure{D}	\measure{A}	\measure{E}	\measure{c}
\end{instrumental}

\vspace{-10pt}Las calles..
\vspace{-10pt}
\begin{instrumental}{}
\measure{25} \measure{26} \measure{27} \measure{28} \measure{29} \measure{30} \measure{31} \measure{32}
\measure{A} \measure{E} \measure{E} \measure{A} \measure{A} \measure{A7;D} \measure{D;A} \measure{E;c}
\measure{I} \measure{V} \measure{V} \measure{I} \measure{I} \measure{I7;IV} \measure{IV;I} \measure{V;c}
\end{instrumental}

\vspace{-10pt}Verbenas..

Het eerste en het tweede couplet vormen samen een zin. Het eerste couplet is het statische gedeelte en het tweede couplet het dynamische gedeelte (de tussendominant truc) dat met de cadans wordt afgesloten. De totale zin ziet er als volgt uit:

\begin{tabular}{*{10}{|p{2em}}}
\multicolumn{4}{c}{Stat.} & \multicolumn{5}{c}{dynamisch} & c\\
\multicolumn{4}{p{12em}@{}}{/\dashfill\textbackslash} & \multicolumn{5}{p{15em}@{}}{/\dashfill\textbackslash}\\
I & V & V & I & I & I7 & IV & I & V & c
\end{tabular}

De functie van de tussendominant I7 (\chord*{A7}) is dat het \chord*{D}-akkoord wordt aangekondigd. Tot die tijd zijn alleen de I en de V (\chord*{A} en \chord*{E}) gebruikt. De \chord*{A7} zegt: let op er gaat iets komen. Bij de \chord*{A7} (eigenlijk al bij de \chord*{A} ervoor) begint het dynamische gedeelte en wordt de zin afgesloten. Er is nu een ook een rustpunt in het nummer (y chulapon).

De progressie van het refrein is dezelfde zin als voor de coupletten is gebruikt alleen wordt deze zin nu twee keer zo snel doorlopen. De eerste twee coupletten duren samen 16 maten. In het refrein wordt dezelfde akkoordenreeks in 8 maten doorlopen. Nu wordt ook de functie van de intro duidelijk. Deze heeft dezelfde statische progressie als het eerste couplet, alleen wordt deze in de intro op het tempo van het refrein doorlopen. De intro fungeert dus als brug (letterlijk met beide verbonden) tussen de twee delen.

\clearpage
\subsection*{Voorbeeld: El Torero}
intro:

E-frygisch

 dynamisch   		 andalusische cadans
/--------------------\/-------------------\
|E		|Am		|G 	F	|E		|
|I		|iv		|III	II	|I		|

statisch
|E	|	|F	|E	|
|I	|	|II	|I	|

Naar A-majeur (E is “toevallig” dominant van A, overgang klinkt dus natuurlijk, gebeurt ook in Julio bij het refrein)

Substitutie van Bm voor D (IV). Bm is de relatieve mineur tegenhanger van D.

einde op vrolijke noot F E (kort uitstapje naar frygisch) A (happy end) in plaats van in frygisch te blijven: G F E (duister einde)
