\section{Zangworkshop Tuna}
Jan Martijn van der Werf, Bas Maas\\
\\
Zingen is eigenlijk een aparte manier van muziek maken: je hebt je instrument altijd bij je. Maar net als met een gewoon muziekinstrument, moet je het juist gebruiken. En met zoiets alledaags als je stem, is dat moeilijker dan je denkt. De gitaar kun je even bijstemmen. Als er een gat in de klankkast zit, dan zie je dat. Met zingen is dat helaas niet zo. Wanneer je de juiste technieken gebruikt, merk je dat je veel meer met je stem kunt dan je zelf denkt. In deze workshop gaan we naar een aantal basisdingen kijken: houding, ademhaling, en stembeheersing.
\subsection*{Houding}
Je lijf is je instrument. Net als met de gitaar is de houding waarin je het instrument bespeelt van belang! Voor zingen geldt: een rechte rug, open klankkast en blaasbalg, en stevig met beide voeten op de aarde staan.
\subsubsection*{Oefening 1: Rechte rug}
\begin{enumerate}
\item Voorover hangen, handen helemaal los. Laat ook je hoofd loshangen. 
\item Kom “wervel voor wervel” overeind. Zet de een recht boven de ander. 
\item Als laatste je kruin recht op de wervelkolom zetten. Merk dat daarmee je voorhoofd een klein beetje naar beneden staat (alsof je kruin aan een touwtje zit)
\end{enumerate}
\subsubsection*{Oefening 2: Neem dat applaus in ontvangst!}
\begin{enumerate}
\item Sta met een rechte rug (zie eventueel oefening 1) 
\item Maak brede schouders
\item Hef je armen omhoog als een Y, met licht geknikte armen (alsof je applaus in ontvangst neemt)
\item Draai je armen naar beneden
\item Neem de “gitaarhouding” aan
\end{enumerate}
\subsubsection*{Oefening 3: Stevig staan: voel de aarde!}
\begin{enumerate}
\item Ga op je tenen staan
\item Ga op je hakken staan
\item Ga op je tenen staan
\item Plof met je hakken op de grond.
\subsubsection*{Oefening 4: Stevig staan: draai een cirkel.}
\begin{enumerate}
\item Als een plank naar voren buigen, totdat je net niet omvalt
\item Draai langzaam een ronde (naar rechts)
\item Doe dit spiraalsgewijs naar binnen tot je recht staat
\end{enumerate}
\subsection*{Adem}
Geluid produceer je door je adem langs de stembanden je mond uit te blazen. Hoe harder je je stembanden aantrekt, hoe hoger het geluid. Met zingen is dat al wat je met je stembanden doet: de hoogte van het geluid bepalen. Zodra je meer met je stembanden (en je strottehoofd) doet, gaat het mis! Dan raken ze overspannen, en kun je nare bijverschijnselen krijgen. Na de repetitie last gehad van je stem? Moet je veel je keel schrapen of kuchen danwel hoesten? Dan gebruik je je stem verkeerd! Dus daarom kijken we naar de basis van zingen: de ademhaling. Wanneer je een huis bouwt, begin je met de fundering, totdat je uiteindelijk bij de zolder uitkomt. Bij zingen is dat niet anders: je ademhaling is het fundament waarop je de rest van je zangbouwwerk neerzet, met de stembanden op zolder...
\subsubsection*{Oefening 5: Inademen}
\begin{enumerate}
\item Ga recht staan (zie oefening 2)
\item Adem zo ver mogelijk uit, totdat er geen lucht meer is, blaas nog verder alle lucht uit.
\item Houd de adem 5sec in.
\item Adem in. Voel wat je buik doet!
\end{enumerate}
\subsubsection*{Oefening 6: Buik en flank}
Met zingen moet je een luchtreservoir hebben, waar je voldoende lucht uit moet kunnen tappen. Denk maar aan een doedelzak: je blaast er lucht in, en vervolgens kun je die luchtzak gebruiken om er lucht uit te kunnen tappen. Dan moet je wel eerst weten waar die luchtzak zit. Deze zit (denkbeeldig) in je buik: Door goed te staan en goed in te ademen, kun je je middenrif (de zwevende ribben) gebruiken om de luchtdoorvoer te reguleren. Daarvoor moet je ruimte maken: als je middenrif naar beneden gaat, moeten je organen ergens naar toe. Daarom zet je je buik uit naar voren, en je flanken naar buiten: dan geef je zoveel mogelijk ruimte om lucht te happen!\\
Waarom geeuwen? Geeuwen zorgt ervoor dat je adamsappel naar beneden zakt (de goede zangpositie!) en dat je mond een mooie holte is, zodat je straks een mooie klank kunt produceren!
\begin{enumerate}
\item Ga recht staan (zie oefening 2)
\item Gaap
\item Houd je mond vast alsof je gaapt, en haal adem met de buik: de buik zet uit naar voren, en je flanken gaan naar buiten
\item Sluit de mond, en houd de adem 30 sec vast
\item Zucht de adem naar buiten 
\end{enumerate}
\subsubsection*{Oefening 7: Cirkel draaien}
\begin{enumerate}
\item Ga recht staan
\item Gaap
\item Houd je mond vast alsof je gaapt, en haal adem met de buik: de buik zet uit naar voren, en je flanken gaan naar buiten.
\item Breng je armen naar boven als een Y.
\item Trek je buik iets aan.
\item Blaas langzaam uit door je buik aan te trekken. Houd je flanken naar buiten. Draai op het tempo van je uitblazen je armen in een cirkel.
\end{enumerate}
\subsubsection*{Oefening 8: Zachtjes een waslijn sissen}
Wanneer je een ballon opblaast, en hem gewoon loslaat, komt de lucht er met een plof uit, en heb je er niets aan. Houd je de uitgang echter strak gespannen, dan komt er gelijkmatig lucht uit, en kun je gecontroleerd tonen maken. Met zingen is dat net zo! Je buik is de ballon, je stembanden zijn de uitgang. 
\begin{enumerate}
\item Ga recht staan
\item Gaap
\item Houd je mond vast alsof je gaapt, en haal adem met de buik: de buik zet uit naar voren, en je flanken gaan naar buiten.
\item Trek je buik iets aan.
\item Adem heel langzaam en zachtjes uit op een sss. Gebruik je buikspier om de adem gelijkmatig uit te blazen. Houd de flanken naar buiten! Zorg dat het een constante waslijn is: een constante uitvoer van lucht. Doe dit 30 sec. Let op je houding!
\end{enumerate}
\subsubsection*{Oefening 9: Lippen ontspannen: brrrr!}
Idem als 8, maar in plaats van een sss op brrr. Zorg ervoor dat je keel en je adamsappel ontspannen zijn!
\subsection*{Stembeheersing}
\subsubsection*{Oefening 10: Brrr naar de kwint}
Oefening 9, maar dan met een kwintsprong. Glijd naar de kwint! En dan weer terug naar de grondtoon. 
\subsubsection*{Oefening 11: Peng}
\begin{lilypond}[quote,fragment,staffsize=26]
  c2 (a4) (bes4) (g'4) (f4)
\end{lilypond}
Ng klank. Gaap.\\
Dan met Ng – i. Let op, geen scherpe i!
\subsubsection*{Oefening 12: toonladder naar de kwint}
Op: Brrr - Oh – Ah
\subsubsection*{Oefening 13: toonladder naar de kwint met octaaf}
Op: Brrr - Oh – Ah
\subsubsection*{Oefening 14: Wie die niet ziet & O Roze Du}
Toonladder naar beneden: \\
\hspace*{3em}Wie die niet ziet, wie die niet ziet en \\
\hspace*{3em}O Rose Du, O Rose Du
\subsubsection*{Oefening 15: tertsladder naar beneden}
\begin{lilypond}[quote,fragment,staffsize=26]
  c4 g'4 e'4 c'4
\end{lilypond}
Op: Brrr - Oh – Ah – Nie
\subsubsection*{Oefening 16: tertsen en secunden}
\begin{lilypond}[quote,fragment,staffsize=26]
  f'4 (a4) (c4) (bes4) (g4) (f4)
\end{lilypond}
\subsubsection*{Oefening 17: Hoe}
\begin{lilypond}[quote,fragment,staffsize=26]
  c4 (a4) b4 (g4) a4 (f4) g4 (e4) f4
\end{lilypond}
