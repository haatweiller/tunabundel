\begin{song}*[ForestGreen]{Spaanse uitspraak}
\vspace{-20pt}
\textbf{Algemene opmerkingen}
\vspace{-12pt}
\begin{compactenum}
\item Zowel klinkers al medeklinkers worden kort voor in de mond uitgesproken.
\item Klemtoon:
\begin{compactenum}
\item Als de laatste letter een medeklinker is maar geen n of s: klemtoon valt op de laatste lettergreep.
\item Als de laatste letter een klinker of een n of s is: klemtoon valt op de \'e\'en na laatste lettergreep.
\item Als het woord een accent bevat, valt de klemtoon op de lettergreep met het accent.
\end{compactenum}
\end{compactenum}

Uitspraak van letters met een andere uitspraak dan in het Nederlands\\
\begin{tabularx}{1.0\textwidth}{l l}
c & klinkt voor en i als th, gelijk het engelse 'thing'. IN de andere gevallen als een k.\\
 & v.b.: cancion (kan'th'i\'on)\\
ch & klinkt als ch net als in het engelse change\\
 & v.b.: chica (tsj\'ieka)\\
d & wordt zacht uitgesproken\\
g & klinkt voor en i als ch in lachen, in alle andere gevallen als de franse g in grand\\
 & v.b.: gerona (cher\'ona)\\
h & is altijd stom\\
j & klinkt als ch in het nederlandse lachen\\
 & v.b.: jota (ch\'otta)\\
ll & klinkt als j\\
 & v.b.: estrella (estr\'eja)\\
\~{n} & klinkt als nj\\
 & v.b.: Espa\~na (Esp\'anja)\\
r & is een harde tong-r, voor in de mond\\
v & klinkt als b\\
 & v.b.: llevar (jeb\'ar)\\
x & klinkt als gs\\
 & v.b.: examen (egs\'amen)\\
z & klinkt als th, gelijk het engelse thing\\
 & v.b.: cerveza ('th'erbe'th'a)\\
\end{tabularx}

\textbf{Klinkers in het algemeen}\\
Elke klinker heeft maar \'e\'en manier om uitgesproken te worden\\
\begin{tabular}{l l}
u & klinkt als oe\\
 & v.b.: Tuna (T\'oena)\\
y & klinkt als i\\
 & v.b.: soy (s\'oj)\\
i & klinkt als ie\\
 & v.b.: sin (s\'ien)\\
\end{tabular}

Tweeklanken bestaan niet, elke letter wordt apart uitgesproken
\end{song}