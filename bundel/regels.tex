\subsection*{Regels van de repetitie}
\footnotesize
 De repetitie zal het volgende formaat hebben:
\begin{itemize}
\item We spelen één optredens set door, om het niveau op pijl te houden. Hierbij gaan we niet te lang stil staan bij liedjes.
\item We spelen één 'nieuw' nummer door dat we nog niet kennen.
\item We herhalen het 'nieuwe' nummer van vorige week, om het als groep goed te kennen.
\end{itemize}

Om te kunnen waarborgen dat alles goed gaat zijn er de volgende regels gemaakt:
\begin{itemize}
\item Men heeft deze muziekbundel bij en een potlood voor aantekening
\item Bij niet leren zijn er de volgende consequenties aanverbonden:
\begin{itemize}
    \item Eerste keer liedje niet kennen, voor in de Peapod het nummer gaan leren terwijl de rest het nummer repeteert.
    \item Tweede keer liedje niet kennen, hoef je de rest van de repetitie niet terug te komen.
    \item Derde keer achterelkaar het liedje niet kennen zal betekenen dat je een nat pak gaat halen in de dichtstbijzijnde fontein.
    \item De niet leer regels zijn tijdens de leden repetitie minder streng voor de foeten. Ze gaan wel op voor de liedjes die de foetenrepetitieleider zegt dat ze moeten kennen.
\end{itemize}
\item Mensen gaan op stem naast elkaar staan zodat men meer op elkaar kan letten. Voor referentie gebruik het overzicht dat Jan Martijn en Bas gemaakt hebben voor Reinier
\item De repetitieleider zal voor het weekend een mail sturen met welke liedjes we spelen de volgende week. Het liefste gebeurt dit op woensdag al.
\end{itemize}
\vfill
\subsection*{Standaard sets}
%\begin{center}
\begin{framed}
	\begin{minipage}[b][105mm][c]{148mm}%
		\hfill\vspace{105mm}%
	\end{minipage}%
\end{framed}
%\end{center}
\normalsize
