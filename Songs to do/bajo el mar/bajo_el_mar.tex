\documentclass[a4paper,11pt]{article}
\usepackage[bw]{tunaChords}

\begin{document}
\begin{song}{Bajo el mar}
\begin{verse}{Intro}
Ariel, escúchame. Ese mundo está muy mal. La vida bajo el mar es mucho 
mejor que el mundo allá arriba.
\end{verse}

\begin{verse}{Couplet 1}
\chord{C}Tú piensas \chord{G7}que en otros lagos \chord{C}las algas más \chord{G7}ver\chord{C}des son\\
\chord{C}Y sueñas \chord{G7}con ir \chord{C}arriba, ¡qué gran \chord{G7}equi\chord{C}vocación!\\
\chord{F}¿No ves que tu \chord{C}propio mundo \chord{G7}no tiene \chord{C}comparación?\\
\chord{F}¿Qué puede \chord{C}haber allá fuera \chord{G7}que causa tal \chord{C}emoción?
\end{verse}

\begin{verse}{Refrein}
\chord{C}Bajo el \chord{F}mar,\\
\chord{C}bajo el \chord{G}mar
\end{verse}

\begin{verse}{Couplet2}
Vives \chord{F}serena, \chord{G}siendo sirena eres \chord{C}feliz\chord{C7}\\
\chord{C}Ellos \chord{F}trabajan sin pa\chord{G7}rar y bajo el \chord{Am}sol para var\chord{D7}iar\\
\chord{F}Mientras nosotros siempre flot\chord{G}amos Bajo el \chord{C}mar
\end{verse}

\begin{verse}{Couplet 3}
\chord{C}Los peces son \chord{G7}muy felices\chord{C}, aquí tienen \chord{G7}liber\chord{C}tad\\
\chord{C}Los peces \chord{G7}allí están \chord{C}tristes, sus casas \chord{G7}son de \chord{C}cristal\\
\chord{F}La vida \chord{C}de aquellos peces \chord{G7}muy larga no \chord{Am}suele \chord{D7}ser\\
\chord{F}Si al dueño \chord{C}le apetece, \chord{G}a mí me van a \chord{C}comer\\
\end{verse}

\repetition{Refrein}

\begin{verse}{Couplet 4}
\chord{C}Nadie nos fríe \chord{F}ni nos coc\chord{G}ina en la sart\chord{C}én \chord{C7}\\
\chord{C}Si no te \chord{F}quieres arriesg\chord{G}ar y los pro\chord{Am}blemas evi\chord{D7}tar\\
\chord{C}Entre \chord{F}burbujas debes que\chord{G}darte\\
\chord{C}Bajo el mar
\end{verse}

\repetition{Refrein}

\begin{verse}{Couplet 5}
\chord{C}Hay siempre \chord{F}ritmo y los senti\chord{G}mos al natu\chord{C}ral\chord{C7}\\
\chord{C}La manta \chord{F}raya va a to\chord{G}car, el esturión \chord{Am}va a \chord{D7}acompañar\\
\chord{C}Tu soló \chord{F}escucha, aqui esta la \chord{G}marcha\\
Bajo el \chord{C}mar
\chord{G}Ya suena la flauta, \chord{C}carpa en el arpa\\
\chord{G7}Al contrabajo ponle atención\\
\chord{F}Oirás las trompetas \chord{C}con el tambor\\
\chord{G}Y el solo de \chord{C}saxofón, sí\\
\chord{F}Con la marimba \chord{C}y el violín\\
\chord{G}Las truchas bailando, \chord{C}el vero cantando\\
\chord{F}Y el pezqueñin, \chord{C}soplando el clarín,\\
\chord{G}Al fin soplará el trom\chord{C}bón
\end{verse}

\repetition{Refrein \textnormal{(Instrumental)}}

\begin{verse}{Couplet 7}
\chord{C}Sí, bajo el \chord{F}mar, \chord{C}bajo el \chord{G}mar\\
\chord{C}Las baila\chord{F}rinas, son las sard\chord{G}inas, ven a \chord{C}bailar\chord{C7}\\
\chord{C}¿Qué mundos \chord{F}quieres explorar si nuestra \chord{Am}banda empieza a to\chord{D7}car?\\
\chord{C}Y las al\chord{F}mejas son casta\chord{G}ñuelas, bajo el \chord{C}mar\\
\chord{C}Y las ba\chord{F}bosas son tan gra\chord{G}ciosas, bajo el \chord{C}mar\\
\chord{C}El cara\chord{F}col es un gran \chord{G}artista\\
\chord{C}Y las bur\chord{Am}bujas llenan la \chord{D7}pista\\
\chord{C}Para que \chord{F}bailes hay buena \chord{G}fiesta
\end{verse}

\begin{verse}{Einde}
\chord{C}Bajo el mar\_\_\_\_\_
\end{verse}
\end{song}
\end{document}